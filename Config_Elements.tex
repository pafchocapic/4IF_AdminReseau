\documentclass[a4paper]{article}

\usepackage[utf8]{inputenc}  
\usepackage[francais]{babel}  
\usepackage[top=2cm, bottom=2cm, left=2cm, right=2cm]{geometry}
\usepackage{graphicx}

\begin{document}

\begin{titlepage}
	~ 
	\vfill
	\begin{center}
		\begin{Huge}
			Projet Administration Réseau : \\ Configuration des éléments de la solution\\
		\end{Huge}
	\vfill
		\textbf{Hexanôme 4211 :} 
			\\Sandra \bsc{Mondain}, Elisa \bsc{Abidh}, 
			\\Gaël \bsc{Motte}, Armand \bsc{Rossius}, 
			\\Nicolas \bsc{Silva}, Julien \bsc{Levesy}\\
	\vfill
	\end{center}
	\vfill
\end{titlepage}

\newpage
\tableofcontents
\newpage

\section{Spécification détaillée de l'architecture}
	\subsection{Définition des matériels}
	Gruck
	
	\subsection{Construction du plan d'adressage}
	Juju
	
\section{Cahier de supervision}
	Armand et Nicolas

%\subsection{Exigences}
%Dans le cadre de ce projet, plusieurs exigences sont mises au premier plan :
%\begin{description}
%\item{Sécurité}\\
%	Il s'agit la de permettre des connexions plus sécurisée et un partage d'informations entre les différentes équipes de l'AIP plus privé. Il s'agit notamment d'éviter que les informations transmises sur les réseaux académiques ne puissent être lues.
%   
%\item{Robustesse}\\
%	Les différents sites de l'AIP doivent être autonome en terme d'équipement et service informatique. Ainsi, si le site de central basé à Lyon tombe, les autres sites doivent rester en mesure de fonctionner. Cela inclut l'identification et authentification, le partage d'informations et l'utilisation des plateformes de TP mobiles.
%   
%\item{Silence}\\
%	Les plateformes de TP utilisent de façon courante le multicast pour partager des informations. Des actions devront être mises en place afin d'éviter que les paquets multicasts émis par ces machines ne soient répliqués par les routeur des réseaux académiques (il s'agit la de leur comportement par défaut).
%
%\end{description}

\end{document}