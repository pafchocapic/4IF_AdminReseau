\documentclass[a4paper]{article}

\usepackage[utf8]{inputenc}  
\usepackage[francais]{babel}  
\usepackage[top=2cm, bottom=2cm, left=2cm, right=2cm]{geometry}
\usepackage{graphicx}

\begin{document}

\begin{titlepage}
	~ 
	\vfill
	\begin{center}
		\begin{Huge}
			Projet Administration Réseau : \\ ÉConfiguration des éléments de la solution\\
		\end{Huge}
	\vfill
		\textbf{Hexanôme 4211 :} 
			\\Sandra \bsc{Mondain}, Elisa \bsc{Abidh}, 
			\\Gaël \bsc{Motte}, Armand \bsc{Rossius}, 
			\\Nicolas \bsc{Silva}, Julien \bsc{Levesy}\\
	\vfill
	\end{center}
	\vfill
\end{titlepage}

\newpage
\tableofcontents
\newpage

%dans ce document, etablir les liste d e tous les poiunts techniques/critiques qui vont etre mis en plaace dans cette archi

\subsection{Exigences}
Dans le cadre de ce projet, plusieurs exigences sont mises au premier plan :
\begin{description}
\item{Sécurité}\\
	Il s'agit la de permettre des connexions plus sécurisée et un partage d'informations entre les différentes équipes de l'AIP plus privé. Il s'agit notamment d'éviter que les informations transmises sur les réseaux académiques ne puissent être lues.
   
\item{Robustesse}\\
	Les différents sites de l'AIP doivent être autonome en terme d'équipement et service informatique. Ainsi, si le site de central basé à Lyon tombe, les autres sites doivent rester en mesure de fonctionner. Cela inclut l'identification et authentification, le partage d'informations et l'utilisation des plateformes de TP mobiles.
   
\item{Silence}\\
	Les plateformes de TP utilisent de façon courante le multicast pour partager des informations. Des actions devront être mises en place afin d'éviter que les paquets multicasts émis par ces machines ne soient répliqués par les routeur des réseaux académiques (il s'agit la de leur comportement par défaut).

\end{description}
	
\subsection{VPN}
\subsubsection{Objectifs}
%permettre la communication site à site
Le VPN est un moyen de communication qui permet de mettre en réseau des sites distants entre eux de manière sécurisé.
Le VPN sera à la base de notre solution. C'est lui qui permettra de répondre aux 3 attentes énoncées précédemment.


%communication possible entre toutes les machines AIP
\subsubsection{Authentification}
Deux niveaux d'identification/authentification seront mis en place pour cette solution.
\begin{description}
\item{Site à site} \\
Une première identification sera effectuée au niveau de chaque routeur d'entrée sur le site de l'AIP. Ce routeur sera équipé un client VPN enbarqué du type Cisco Asa. Ces routeurs permettront l'accès au deux réseaux : 
\begin{itemize}
\item Le réseau académique du site
\item Le réseau privé virtuel de l'AIP dans sa blobalité
\end{itemize}
Cette connection VPN se faisant site à site, des passphrase globale ou, mieux, des certificats devront être mis en place pour assurer une double authentification.

\item{Personelle} \\
Cette identification porte au niveaux des utilisateurs eux-mêmes. Il s'agit d'ajouter les utilisateurs de l'AIP au groupe des utilisateurs AIP dans les LDAP des université d'acceuil.\\
Seule les personnes appartenant à ce groupe pourront se logger sur les machines de l'AIP.

\end{description}

\subsection{Cas des plateformes mobiles}
\subsubsection{Communications entre automates}
Les automates doivent pouvoir communiquer entre eux pour le partage des variables globales

\begin{itemize}
	\item multicast
	\item unicast
\end{itemize}

\subsubsection{Intégration aux réseaux}
un routeur deja prêts pour les acceuilir, chaque atelier est en mesure de les acceuilir sans installation de matériel supplémentaire.\\
 ajout de routeur configuré pour non replication des broadcasts udp\\


\subsection{Impacts sur les réseaux existants}
recablage complet de TOUS les atelier machines AIP\\
utlisation des ldap existants


\end{document}