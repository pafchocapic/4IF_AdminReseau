\documentclass[a4paper]{article}

\usepackage[utf8]{inputenc}  
\usepackage[francais]{babel}  
\usepackage[top=2cm, bottom=2cm, left=2cm, right=2cm]{geometry}
\usepackage{graphicx}

\begin{document}

\begin{titlepage}
	~ 
	\vfill
	\begin{center}
		\begin{Huge}
			Projet Administration Réseau : \\ Spécifications de l'architecture DNS et Tutorial de maintenance de l'installation\\
		\end{Huge}
	\vfill
		\textbf{Hexanôme 4211 :} 
			\\Sandra \bsc{Mondain}, Elisa \bsc{Abidh}, 
			\\Gaël \bsc{Motte}, Armand \bsc{Rossius}, 
			\\Nicolas \bsc{Silva}, Julien \bsc{Levesy}\\
	\vfill
	\end{center}
	\vfill
\end{titlepage}

\newpage
\tableofcontents
\newpage

\section{Spécification de l'architecture DNS}

\subsection{Politique de nommage et déploiement des services d'annuaires machines }



\subsection{Configuration de base des serveurs DHCP}

% TODO, vérifier si il est possible de scanner les ports d'emissions des paquets de façon à déterminer
% si la requette DHCPREQUEST proviens d'une machine ou d'un automate...

\subsubsection{Site Lyon : AIP Bat Jacquard }

Plages IP addressables : 
\begin{itemize}
\item[10.1.1.2 à 10.1.1.253]
\item[10.1.2.2 à 10.1.2.253]
\end{itemize}

\subsubsection{Site Lyon : AIP Bat Ferrié}

Plages IP addressables : 
\begin{itemize}
\item[10.2.1.2 à 10.2.1.253]
\item[10.2.2.2 à 10.2.2.253]
\end{itemize}

\subsubsection{Site Roanne : IUT}

Plages IP addressables : 
\begin{itemize}
\item[10.3.1.2 à 10.3.1.253]
\item[10.3.2.2 à 10.3.2.253]
\end{itemize}

\subsection{Gestion d'annuaires utilisateur LDAP}

Afin de garantir une qualité de service et une utilisabilté même si le site de Lyon-Jacquard est injoignable par réseau, nous décidons d'appliquer une politique de réplication d'un groupe LDAP d'utilisateurs autorisés à utiliser les équipements de l'AIP, sur les trois annuaires LDAP actuellement en fonction sur les campus équpipés d'installations AIP.

\section{Tutoriel de maintenance}

\subsection{Comment ajouter une machine type P.C sur le réseau ?}

\subsubsection{Restauration d'une configuration standard sur le poste}

\subsubsection{Configuration du poste selon contexte}

\subsubsection{Création d'une entrée dans l'annuaire DNS}

\subsubsection{Création d'une entrée dans l'annuaire LDAP du campus}

\subsubsection{Bind de la machine au LDAP du campus}

\subsection{Comment supprimer une machine type P.C sur le réseau ?}

\subsubsection{Suppression de l'entrée LDAP de la machine}

\subsubsection{Suppression de l'entrée DNS de la machine}

\subsection{Comment déplacer une plateforme de TP temporairement ?}

\subsubsection{Sur le site de départ}

\subsubsection{Sur le site de déplacement}

\subsection{Comment déplacer une plateforme de TP définitivement ?}

\subsubsection{Sur le site de départ}

\subsubsection{Sur le site de déplacement}

\subsection{Comment ajouter une plateforme de TP  ?}

\subsubsection{Création de l'entrée DNS}

\subsection{Comment supprimer une plateforme de TP définitivement ?}

\subsubsection{Suppression de l'entrée DNS}

\end{document}