\documentclass[a4paper]{article}

\usepackage[utf8]{inputenc}  
\usepackage[francais]{babel}  
\usepackage[top=2cm, bottom=2cm, left=2cm, right=2cm]{geometry}
\usepackage{graphicx}

\begin{document}

\begin{titlepage}
	~ 
	\vfill
	\begin{center}
		\begin{Huge}
			Projet Administration Réseau : \\ Étude d'architecture\\
		\end{Huge}
	\vfill
		\textbf{Hexanôme 4211 :} 
			\\Sandra \bsc{Mondain}, Elisa \bsc{Abidh}, 
			\\Gaël \bsc{Motte}, Armand \bsc{Rossius}, 
			\\Nicolas \bsc{Silva}, Julien \bsc{Levesy}\\
	\vfill
	\end{center}
	\vfill
\end{titlepage}

\newpage
\tableofcontents
\newpage

\section{Architecture}

	\subsection{Description} % redétailler en séparant interconnexion et sur chaque site. mettre en avant la sécurité. 	
	
	Afin de mettre en place une communication sécurisée entre les différents sites, nous optons pour une solution basée sur un VPN (Virtual Private Network). 
	
	\paragraph*{} % Routeurs VPN
	L'utilisation du VPN ne concernant que les machines et les automates de l'AIP, nous plaçons sur chaque site de l'AIP un routeur équipé d'un client VPN qui fait la passerelle entre les automates et le reste du réseau local. 
	
	\paragraph*{} % Switchs VLAN
	Comme les automates peuvent être situés dans différentes salles sur chaque site, on mettra en place un sous-réseau par salle, en utilisant des switchs VLAN (Virtual Local Area Network). L'utilisation de ces switchs permet également de limiter le broadcast UDP des automates sur le réseau. 
	
	\paragraph*{} % DNS
	Nous allons mettre en place un serveur DNS par site. +1 phrase de Juju pour compléter. 
	
	\paragraph*{} % Authentification et Annuaires LDAP
	L'authentification se fera au moyen d'un groupe d'utilisateur répliqué sur tous les annuaires LDAP des différents campus. Ainsi, les utilisateurs mobiles d'un site AIP à l'autre pourront s'authentifier où qu'ils soient. 
	
	\paragraph*{} % Explication du schéma
	Sur le schéma ci-dessous, on peut voir en gris les différents éléments qui constituent le réseau actuellement. Les nouveaux composants sont représentés en vert : un switch au niveau de chaque salle et un routeur VPN au niveau de chaque site. Le tunnel VPN ainsi créé est représenté en bleu. 
	
	\paragraph*{} % Schéma
	TODO : scanner le schéma de la séance 1 et l'insérer ici =D 

	
\section{Plans de nommage et d'adressage}
	\subsection{Plan de nommage}

	Pour rappel, les exigences au niveau du plan de nommage sont :\\
	\begin{itemize}
	\item Gestion des instituts
	\item Gestion des sites présents dans les instituts
	\item Type l'entrée
	\item Générique et extensible
	\end{itemize}	

	Afin de répondre aux différentes exigences de nommage des machines nous proposons le plan de nommage suivant.	
	
	\textbf{Institut-Site-Type-Nom}	
	
	\begin{description}
	\item[Institut]\hfill\\
	Il s'agit ici d'un acronyme sur 4 caractères désignant l'institution possédant les ateliers AIP.\\
	Les institutions actuellement présentes sont :  INSA - Roan - Sten
	
	\item[Site]\hfill\\
	Nom de site commençant par les lettres "AIP", puis le nom du batiment, puis le numéro de l'atelier si il existe plusieurs ateliers par batiments.
	
	\item[Type]\hfill\\
	Type de la machine reliée au réseau, soit A "Automate", D "DNS", Sn-x"Salle n + numero du poste par salle x"	
	
	\item[Nom]\hfill\\
	Nom de la machine par type, numéro unique	
	\end{description}	
	
	\textit{Exemples : INSA-AIPJacquard-A-02, ROAN-AIPMachin2-C-04, STET-AIPBidule3-S3-02}	
	

	\subsection{Plan d'adressage}

	
\end{document}
