\documentclass[a4paper]{article}

\usepackage[utf8]{inputenc}  
\usepackage[francais]{babel}  
\usepackage[top=2cm, bottom=2cm, left=2cm, right=2cm]{geometry}
\usepackage{graphicx}

\begin{document}

\begin{titlepage}
	~ 
	\vfill
	\begin{center}
		\begin{Huge}
			Projet Administration Réseau : \\ Étude d'architecture\\
		\end{Huge}
	\vfill
		\textbf{Hexanôme 4211 :} 
			\\Sandra \bsc{Mondain}, Elisa \bsc{Abidh}, 
			\\Gaël \bsc{Motte}, Armand \bsc{Rossius}, 
			\\Nicolas \bsc{Silva}, Julien \bsc{Levesy}\\
	\vfill
	\end{center}
	\vfill
\end{titlepage}

\newpage
\tableofcontents
\newpage

\section{Architecture}

	\subsection{Description} % redétailler en séparant interconnexion et sur chaque site. mettre en avant la sécurité. 	
	
	Afin de mettre en place une communication sécurisée entre les différents sites, nous optons pour une solution basée sur un VPN (Virtual Private Network). 
	
	\paragraph*{} % Routeurs VPN
	L'utilisation du VPN ne concernant que les machines et les automates de l'AIP, nous plaçons sur chaque site un routeur VPN qui fait la passerelle entre les automates et le reste du réseau local. Comme les automates peuvent être situés dans différentes salles sur chaque site, on mettra en place un sous-réseau par salle si ce n'est pas déjà le cas. 
	
	\paragraph*{} % DNS
	
	
	\paragraph*{} % Authentification et Annuaires LDAP
		Deux niveaux d'identification/authentification seront mis en place pour cette solution.
		\begin{description}
		\item{Site à site} \\
		Une première identification sera effectuée au niveau de chaque routeur d'entrée sur le site de l'AIP. Ce routeur sera équipé un client VPN enbarqué du type Cisco Asa. Ces routeurs permettront l'accès au deux réseaux : 
		\begin{itemize}
		\item Le réseau académique du site
		\item Le réseau privé virtuel de l'AIP dans sa blobalité
		\end{itemize}
		Cette connection VPN se faisant site à site, des passphrase globale ou, mieux, des certificats devront être mis en place pour assurer une double authentification.
		\item{Personelle} \\
		Cette identification porte au niveaux des utilisateurs eux-mêmes. Il s'agit d'ajouter les utilisateurs de l'AIP au groupe des utilisateurs AIP dans les LDAP des université d'acceuil.\\
		Seule les personnes appartenant à ce groupe pourront se logger sur les machines de l'AIP.
		\end{description}	
	
	\paragraph*{} % Explication du schéma
	Sur le schéma ci-dessous, on peut voir en gris les différents éléments qui constituent le réseau actuellement. Les nouveaux composants sont représentés en vert : un routeur au niveau de chaque salle et un routeur VPN au niveau de chaque site (reliant les différents routeurs des sous-réseaux des salles). Le tunnel VPN ainsi créé est représenté en bleu. 
	
	\paragraph*{} % Schéma
	TODO : scanner le schéma de la séance 1 et l'insérer ici =D 

	
\section{Plans de nommage et d'adressage}
	\subsection{Plan de nommage}

	Pour rappel, les éxigences au niveau du plan de nommage sont :\\
	\begin{itemize}
	\item Gestion des instituts
	\item Gestion des sites présents dans les instituts
	\item Type l'entrée
	\item Générique et extensible
	\end{itemize}	

	Afin de répondre aux différentes éxigences de nommage des machines nous proposons le plan de nommage suivant.	
	
	\textbf{Institut-Site-Type-Nom}	
	
	\begin{description}
	\item[Institut]\hfill\\
	Il s'agit ici d'un acronyme sur 4 caractères désignant l'institution possédant les ateliers AIP.\\
	Les institutions actuellement présentes sont :  INSA - Roan - Sten
	
	\item[Site]\hfill\\
	Nom de site commençant par les lettres "AIP", puis le nom du batiment, puis le numéro de l'atelier si il existe plusieurs ateliers par batiments.
	
	\item[Type]\hfill\\
	Type de la machine reliée au réseau, soit A "Automate", D "DNS", Sn-x"Salle n + numero du poste par salle x"	
	
	\item[Nom]\hfill\\
	Nom de la machine par type, numéro unique	
	\end{description}	
	
	\textit{Exemples : INSA-AIPJacquard-A-02, ROAN-AIPMachin2-C-04, STET-AIPBidule3-S3-02}	
	

	\subsection{Plan d'adressage}

	
\end{document}
