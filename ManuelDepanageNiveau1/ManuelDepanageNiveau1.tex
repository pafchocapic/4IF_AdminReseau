\documentclass[a4paper]{article}

\usepackage[utf8]{inputenc}  
\usepackage[francais]{babel}  
\usepackage[top=2cm, bottom=2cm, left=2cm, right=2cm]{geometry}
\usepackage{graphicx}

\begin{document}

\begin{titlepage}
	~ 
	\vfill
	\begin{center}
		\begin{Huge}
			Projet Administration Réseau : \\ Manuel de dépannage Niveau 1\\
		\end{Huge}
	\vfill
		\textbf{Hexanôme 4211 :} 
			\\Sandra \bsc{Mondain}, Elisa \bsc{Abidh}, 
			\\Gaël \bsc{Motte}, Armand \bsc{Rossius}, 
			\\Nicolas \bsc{Silva}, Julien \bsc{Levesy}\\
	\vfill
	\end{center}
	\vfill
\end{titlepage}

\newpage
\tableofcontents
\newpage


\section{Introduction}
\subsection{Objectifs}
%expliquer à l'utilisateur de base qu'il devra assurer un support de premier niveau, simple et guidé
Ce document constituera la base de connaissance minimale pour permettre au personnel travaillant sur les plateformes de l'AIP d'assurer un dépannage de niveau 1.\\
Il ne requiert aucune compétence ni connaissance spécifique. Tous les élements de compréhension seront inclus dans ce document.
\subsection{Moyens}
%description des outils qu'il devra utiliser,
%si mise en place de commandes, les décrire ici avant
Afin de permettre ce support, certains outils et utilitaires devront être utilisés. Ceux-ci sont généralement accessible sur les machines de l'AIP.\\
Dans cette section, nous décrirons brièvement ces outils, comment les activer ou y accéder.

\subsubsection{La lampe de poche}
Bien que simple, elle peut vite devenir indispensable.

\subsubsection{L'invite de commande}
Il s'agit un environnement dans lequel l'utilisateur est invité à saisir des commandes sous formes de texte.
Il ne sera pas nécessaire de connaitre les commandes à saisir, elle seront décrite en temps utile.\\
Pour y accéder :
\begin{enumerate}
	\item Menu Démarrer
	\item Exécuter
	\item Saisir CMD
	\item appuyer sur ok
\end{enumerate}
\subsubsection{Commandes courantes}
Certaines commandes sont à la base de bon nombre de manipulation, parmi celles-ci : 
\begin{itemize}
\item echo : \\
\item IpConfig
Permet d'envoyer des messages sur le réseaux vers des sockets.  
\item ipconfig (ou ifconfig pour linux)\\
Cette commande permet de récupérer les informations de configuration de la machine : \\
\begin{itemize}
\item[•]Identification physique : liste des interfaces, adresses MAC et type de matériel.
\item[•]Identification logique : nom de la machine, adresses IP.
\item[•]Identification des réseaux : adresses IP et masque.
\item[•]Identification des machines assurant un service particulier (annuaire, routage, ...).
\end{itemize}
\item ping \\
On effectue cette commande sur une adresse IP soit sur un nom de machine 
\item nslookup

\end{itemize}

\subsubsection{Client VPN}
Les clients VPN sont des utilitaires qui permettent à des machines distantes d'être virtuellement connectées au réseau de l'AIP.\\
Normalement, ces clients sont installés de bases sur la machines mises à dispositions des membres de l'AIP, toutefois, il est possible de les télécharger. Cette étape ne sera pas détaillé ici.

\subsubsection{Autre Utilitaire}
Voir ce qu'on peut mettre ici


\section{Protocole opératoire}
% montrer qu'il devra suivre un protocol sous forme d'arbre, le guider autant que possible
%a chaque observation, on elague une partie de l'arbre, et on le guide vers la suite.

%lui expliquer également que si à la fin (et à la fin uniquement) de ces procédures, sont problème reste sans réponse, il doit contacter les admins à lyon.
Comme expliqué en introduction, certaines interventions devront être effectuées par les utilisateurs courants de l'AIP.\\
Afin de simplifier ces interventions, une procédure va être mise en place. Toutes se dérouleront sur le même principe : 
\begin{enumerate}
	\item \textbf{Détermination de la plateforme impactée} Il s'agit la de déterminer quelle segment de l'installation est en défaut. Pour cela, un certain nombre d'étape simple seront effectuées pour obtenir les éléments de décisions.
	\item \textbf{Recherche de solution} En fonctions des éléments relevés dans l'étape précédente, un redirections vers la section de solution concernée sera effectuée.
	\item \textbf{Nouvelle tentative} Une nouvelle tentative doit être effectuée. Si le problème persiste, réessayer de suivre cette procédure de nouveau.
	\item \textbf{Appel au central} En cas d'absence de solution en suivant les procédures définies dans ce document, l'appel au central de Lyon peut être une solution.
\end{enumerate}

\section{Détermination de la plateforme impactée}
Afin de déterminer quelle est la portions de l'installation impactée par la panne, une journée type sera retracée. Essayer de répéter les manipulations comme décrite jusqu'à apparition du problème.\\
Trois utilisations classiques sont prévues : 
\begin{itemize}
	\item \textbf{Utilisation d'un PC}
	\item \textbf{Utilisation d'une plateforme de TP mobile}
	\item \textbf{Utilisation d'un laptop}
\end{itemize}

Il est important de suivre les manipulations dans l'ordre où elles sont décrite, car même si la panne est visible sur une machine, elle vient peut être d'une autre installation.


\subsection{Utilisation d'un PC}
\subsubsection{Problème à l'arrivée sur la machine}
L'écran est éteint, la machine ne démarre pas etc

\subsubsection{Problème pour se connecter sur la machine}
A l'entrée des identifiants, la machine refuse le couple identifiant/mot de passe.

\subsubsection{Impossible d'accéder à internet \& mails}
Aucune page web ne se charge, les mails ne sont jamais recus ni envoyés.

\subsubsection{Problème liés à l'utilisation d'application avec les plateformes de TP mobiles}
Un TP requiert que l'on se connecte depuis un des PC de l'AIP et utilise une application spécifique à ce TP. Elle indique des erreur.
		
\subsection{Utilisation d'une plateforme de TP mobile}
\subsubsection{La plateforme de TP semble inopérante}

\subsubsection{La plateforme de TP indique des problèmes réseaux}

\subsubsection{Les paramètres globaux ne sont pas partagés entre différents sites}
	%je me pose sur une plateforme de TP
		%elle ne démarre pas
		%elle affiche des erreur de réseau
		%elle m'insulte en allemand

\subsection{Utilisation d'un laptop}

\subsubsection{Le laptop ne parvient pas à dialoguer avec les serveurs de l'AIP}
	%J'utilise mon laptop
		
% après voir ecrit cette journée type et les différentes erreurs, il est pratiquement toujours possible de rediriger vers l'une ou l'autre des section suivantes.

\section{Dépannage d'un PC de gestion de l'AIP}
%on ecrit une subsection par etape de dépannage
%l'idée est qu'une fois que le mec à suivi la subsection 1, il passe à la subsection 2 si ca ne marche toujours pas.

\section{Dépannage d'une Plateforme de TP Mobile}
%on ecrit une subsection par etape de dépannage
%l'idée est qu'une fois que le mec à suivi la subsection 1, il passe à la subsection 2 si ca ne marche toujours pas.

\section{Dépannage d'un PC personnel et laptop}
%on ecrit une subsection par etape de dépannage
%l'idée est qu'une fois que le mec à suivi la subsection 1, il passe à la subsection 2 si ca ne marche toujours pas.





\end{document}
