\documentclass[a4paper]{article}

\usepackage[utf8]{inputenc}  
\usepackage[francais]{babel}  
\usepackage[top=2cm, bottom=2cm, left=2cm, right=2cm]{geometry}
\usepackage{graphicx}

\newcounter{SolutionCounter}[subsubsection]
\newcounter{SubSolutionCounter}[SolutionCounter]
\newcommand{\soluce}{\stepcounter{SolutionCounter} \paragraph{Solution \arabic{SolutionCounter} : } }
\newcommand{\subsoluce}{\stepcounter{SubSolutionCounter} \\\textbf{Solution  \arabic{SolutionCounter}.\arabic{SubSolutionCounter} : } }

\begin{document}

\begin{titlepage}
	~ 
	\vfill
	\begin{center}
		\begin{Huge}
			Projet Administration Réseau : \\ Manuel de dépannage Niveau 1\\
		\end{Huge}
	\vfill
		\textbf{Hexanôme 4211 :} 
			\\Sandra \bsc{Mondain}, Elisa \bsc{Abidh}, 
			\\Gaël \bsc{Motte}, Armand \bsc{Rossius}, 
			\\Nicolas \bsc{Silva}, Julien \bsc{Levesy}\\
	\vfill
	\end{center}
	\vfill
\end{titlepage}

\newpage
\tableofcontents
\newpage


\section{Introduction}
\subsection{Objectifs}
%expliquer à l'utilisateur de base qu'il devra assurer un support de premier niveau, simple et guidé
Ce document constituera la base de connaissance minimale pour permettre au personnel travaillant sur les plateformes de l'AIP d'assurer un dépannage de niveau 1.\\
Il ne requiert aucune compétence ni connaissance spécifique. Tous les élements de compréhension seront inclus dans ce document.
\subsection{Moyens}
%description des outils qu'il devra utiliser,
%si mise en place de commandes, les décrire ici avant
Afin de permettre ce support, certains outils et utilitaires devront être utilisés. Ceux-ci sont généralement accessibles sur les machines de l'AIP.\\
Dans cette section, nous décrirons brièvement ces outils, comment les activer ou y accéder.

\subsubsection{La lampe de poche}
Bien que simple, elle peut vite devenir indispensable.

\subsubsection{L'invite de commande}
Il s'agit d'une fenêtre dans laquelle l'utilisateur est invité à saisir des commandes sous forme de texte.
Il ne sera pas nécessaire de connaître les commandes à saisir, elle seront décrites en temps utile.\\
Pour y accéder :
\begin{enumerate}
	\item Menu Démarrer
	\item Exécuter
	\item Saisir : "CMD"
	\item Appuyer sur \verb![OK]!
\end{enumerate}
\subsubsection{Commandes courantes}
Certaines commandes sont à la base de bon nombre de manipulations, parmi celles-ci : 
	\begin{itemize}
	\item \texttt{echo} : permet de renvoyer des caractères au système source.
		
	\item \texttt{ipconfig} (ou \texttt{ifconfig} sous Linux) : \\
	Cette commande permet de récupérer les informations de configuration de la machine : 
		\begin{itemize}
		\item[•]Identification physique : liste des interfaces, adresses MAC et type de matériel.
		\item[•]Identification logique : nom de la machine, adresses IP.
		\item[•]Identification des réseaux : adresses IP et masque.
		\item[•]Identification des machines assurant un service particulier (annuaire, routage, etc).
		\end{itemize}
		
	\item \texttt{ping} : \\
	On effectue cette commande sur une adresse IP ou sur un nom de machine. Ceci permet de savoir : 
		\begin{itemize}	
		\item[•]Si une machine est atteignable
		\item[•]Quel délai moyen peut être envisagé pour atteindre cette machine
		\end{itemize}
	Le ping sur un nom machine permet de savoir si le poste source peut atteindre le DNS et si ce service est bien actif.
	Exemple d'utilisation : ping 127.0.0.1 (votre ordinateur).
	
	\item \texttt{tracert} (ou \texttt{traceroute} sous Linux) : \\
	Cette commande permet d'identifier les routes suivies par les paquets pour atteindre une cible. Ceci permet de pouvoir reconstruire la topologie du réseau.
	Exemple d'utilisation : tracert google.fr
	
	\item \texttt{netstat} :\\
	La commande netstat est une commande affichant des informations sur les connexions réseau, les tables de routage et un certain nombre de statistiques. 
	Ces informations peuvent être présenté par  protocole (IP, TCP, UDP, ICMP : netsat -s), être relatives à une interface réseau (netstat -e), aux tables de routage (netstat -r) 
	ou aux connexions actives (netstat -a). D'autres options permettent un filtrage sur les connexions actives (-n = alla active connections, -p = TCP conections, -p UDP = UDP connections).
	
	\item \texttt{nslookup} : \\
	La commande nslookup permet d'interroger les serveurs DNS pour obtenir les informations définies pour un domaine déterminé.
	\end{itemize}


\subsubsection{Le client VPN}
Les clients VPN sont des utilitaires qui permettent à des machines distantes d'être virtuellement connectées au réseau de l'AIP.\\
Normalement, ce client est installé de base sur la machine mise à disposition des membres de l'AIP. Il est également possible de le télécharger, toutefois, cette étape ne sera pas détaillée ici.


\section{Protocole opératoire}
Comme expliqué en introduction, certaines interventions devront être effectuées par les utilisateurs courants de l'AIP.\\
Afin de simplifier ces interventions, une procédure va être mise en place. Toutes se dérouleront sur le même principe : 
\begin{enumerate}
	\item \textbf{Détermination de la plateforme impactée} Il s'agit là de déterminer quel segment de l'installation est en défaut. Pour cela, un certain nombre d'étapes simples seront effectuées pour obtenir les éléments de décision.
	\item \textbf{Recherche de solution} En fonction des éléments relevés dans l'étape précédente, une redirection vers la section de solution concernée sera effectuée.
	\item \textbf{Nouvelle tentative} Une nouvelle tentative doit être effectuée. Si le problème persiste, essayez de suivre cette procédure de nouveau.
	\item \textbf{Appel au central} En cas d'absence de solution en suivant les procédures définies dans ce document, l'appel au central de Lyon peut être une solution.
\end{enumerate}

\section{Détermination de la plateforme impactée}
Afin de déterminer quelle est la portion de l'installation impactée par la panne, une journée type sera retracée. Essayer de répéter les manipulations comme décrites jusqu'à apparition du problème.\\
Trois utilisations classiques sont prévues : 
\begin{itemize}
	\item \textbf{Utilisation d'un PC}
	\item \textbf{Utilisation d'une plateforme de TP mobile}
	\item \textbf{Utilisation d'un ordinateur portable}
\end{itemize}

Il est important de suivre les manipulations dans l'ordre où elles sont décrites, car même si la panne est visible sur une machine, elle vient peut être d'une autre installation.


\subsection{Utilisation d'un PC}
\subsubsection{Problème à l'arrivée sur la machine}
L'écran est éteint, la machine ne démarre pas, etc. \\

\soluce Vérifiez que toutes les installations physiques de l'ordinateur sont bien en place (câbles connectés, alimentation électrique disponible, etc). 
\soluce Essayez d'intervertir une partie du matériel pour vérifier que l'écran fonctionne sur une autre tour, et inversement, que la tour fonctionne sur un autre écran. 
\subsoluce Dans le cas où l'un des matériels se montre défectueux, adressez-vous à la DSI pour remplacement du matériel en question.
\subsoluce Si cela ne fonctionne toujours pas, contacter directement la DSI

\subsubsection{Problèmes pour se connecter sur la machine}
À l'entrée des identifiants, la machine refuse le couple identifiant/mot de passe.
\soluce Essayez d'autres identifiants, ceux d'un collègue par exemple. 
\subsoluce Votre collègue peut s'identifier, dans ce cas, vous ne disposez pas des accréditations nécessaires pour vous identifier au sein de l'AIP, veuillez vous référer aux administrateurs.
\subsoluce Votre collègue ne peut pas s'identifier non plus. Vérifiez la connexion des câbles Ethernet et réessayez. Si le problème persiste, référez-vous aux administrateurs.


\subsubsection{Impossible d'accéder à internet \& emails}
Aucune page web ne se charge, les emails ne sont jamais reçus ni envoyés.

\soluce Vérifiez la connexion des câbles Ethernet. 
\soluce Demandez à vos collègues si ils ont bien accès à internet. 
\subsoluce Si ils n'ont pas plus accès à internet, veuillez contacter les administrateurs.
\subsoluce Si ils ont internet, l'utilisation de l'invite de commande avec "ping" et "nslookup" peut fournir des informations.
\begin{itemize}
	\item ping google.com -> si les paquets sont transmis, vérifier l'état de l'option "travail hors connexion" d'Internet Explorer.
	\item nslookup -> Si l'adresse indiquée n'est pas celle indiqué sur le plan d'adressage de votre atelier, veuillez contacter les administrateurs.
\end{itemize}

\subsubsection{Problèmes liés à l'utilisation d'applications avec les plateformes de TP mobiles}
Un TP requiert que l'on se connecte depuis un des PC de l'AIP et utilise une application spécifique à ce TP. Elle indique des erreurs.

\soluce Veuillez vérifier la connexion du/des câble(s) Ethernet de la plateforme de TP.
\soluce Tentez d'accéder à internet (affichage d'une page web par exemple). En cas de problème, veuillez vous reporter à la section précédente.
\soluce Demandez à un de vos collègue d'effectuer la même manipulation sur une autre machine.
\subsoluce Si il parvient à effectuer la manipulation, veuillez vous reporter à la section précédente.
\subsoluce Si il ne parvient pas à effectuer la manipulation non plus, veuillez contacter les administrateurs.
		
\subsection{Utilisation d'une plateforme de TP mobile}
\subsubsection{La plateforme de TP semble inopérante}
Les tableaux de contrôles de la plateforme de TP semblent inactifs, les commandes ne répondent pas. 

\soluce Vérifiez la connexion des câbles d'alimentation de la plateforme de TP.
\soluce Reportez le TP et faites appel aux administrateurs.

\subsubsection{La plateforme de TP indique des problèmes réseaux}
\soluce Vérifiez la connexion des câbles Ethernet de la plateforme de TP concernée
\soluce Vérifiez que d'autres plateformes de présentent pas le même défaut
\subsoluce Si le problème est visible sur d'autres machines, veuillez contacter les administrateurs.


\subsubsection{Les paramètres globaux ne sont pas partagés entre différents sites}
La plateforme de TP indique que des données globales partagées entre plusieurs plateformes ne sont pas disponibles.

\soluce Vérifiez que toutes les machines nécessaires sont allumées.
\soluce Vérifiez que ces machines n'affichent pas d'erreur, dans ce cas, reportez pas vous à la sous-section précédente.
\soluce Si le problème persiste, veuillez contacter les administrateurs.

\subsection{Utilisation d'un ordinateur portable}

\subsubsection{L'ordinateur portable ne parvient pas à dialoguer avec les serveurs de l'AIP}
Il est impossible d'utiliser les applications ou récuperer les documents à partir des serveurs de l'AIP. 

\soluce Vérifiez que l'ordinateur en question possède bien un accès à internet, en ouvrant une page web par exemple.
\soluce Vérifiez la disponibilité du VPN Client sur la machine. 
\subsoluce Dans la barre de tâches de Windows, vérifiez la présence de l'icône CISCO. 
\subsoluce Réinstallez le client VPN.


\end{document}
