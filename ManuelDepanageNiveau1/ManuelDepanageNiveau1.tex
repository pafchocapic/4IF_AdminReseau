\documentclass[a4paper]{article}

\usepackage[utf8]{inputenc}  
\usepackage[francais]{babel}  
\usepackage[top=2cm, bottom=2cm, left=2cm, right=2cm]{geometry}
\usepackage{graphicx}

\begin{document}

\begin{titlepage}
	~ 
	\vfill
	\begin{center}
		\begin{Huge}
			Projet Administration Réseau : \\ Manuel de dépannage Niveau 1\\
		\end{Huge}
	\vfill
		\textbf{Hexanôme 4211 :} 
			\\Sandra \bsc{Mondain}, Elisa \bsc{Abidh}, 
			\\Gaël \bsc{Motte}, Armand \bsc{Rossius}, 
			\\Nicolas \bsc{Silva}, Julien \bsc{Levesy}\\
	\vfill
	\end{center}
	\vfill
\end{titlepage}

\newpage
\tableofcontents
\newpage


\section{Introduction}
\subsection{Objectifs}
%expliquer à l'utilisateur de base qu'il devra assurer un support de premier niveau, simple et guidé
Ce document constituera la base de connaissance minimale pour permettre au personnel travaillant sur les plateformes de l'AIP d'assurer un dépannage de niveau 1.\\
Il ne requiert aucune compétence ni connaissance spécifique. Tous les élements de compréhension seront inclus dans ce document.
\subsection{Moyens}
%description des outils qu'il devra utiliser,
%si mise en place de commandes, les décrire ici avant
Afin de permettre ce support, certains outils et utilitaires devront être utilisés. Ceux-ci sont généralement accessible sur les machines de l'AIP.\\
Dans cette section, nous décrirons brièvement ces outils, comment les activer ou y accéder.

\subsubsection{La lampe de poche}
La sortir de sa poche-> appuyer sur le bouton
\subsubsection{L'invite de commande}
Il s'agit un environnement dans lequel l'utilisateur est invité à saisir des commandes sous formes de texte.
Il ne sera pas nécessaire de connaitre les commandes à saisir, elle seront décrite en temps utile.\\
Pour y accéder :
\begin{enumerate
	\item Menu Démarrer
	\item Exécuter
	\item Saisir CMD
	\item apuyer sur ok
\end{enumerate}
\subsubsection{Commandes courantes}
Certaines commandes sont à la base de bon nombre de manipulation, parmis celles-ci : 
\begin{itemize}
ZAZA
\item IpConfig
\item Ping
\item nslookup
\end{itemize}


\section{Protocole opératoire}
% montrer qu'il devra suivre un protocol sous forme d'arbre, le guider autant que possible
%a chaque observation, on elague une partie de l'arbre, et on le guide vers la suite.

%lui expliquer également que si à la fin (et à la fin uniquement) de ces procédures, sont problème reste sans réponse, il doit contacter les admins à lyon.
Comme expliqué en introduction, les interventions...

\section{Détermination de la plateforme impactée}
%trouver ici tous les éléments qui peuvent éventuellement exploser
%
%le plus simple je pense est de reprendre le déroulement d'une journée normale de travail pour un membre de l'AIP

%exemple:



\subsection{Utilisation d'un PC}
	%je me pose devant un PC de gestion
		%ecran tout noir
		%impossible de me log
		%impossible d'acceder à internet
		
\subsection{Utilisation d'une plateforme de TP mobile}
	%je me pose sur une plateforme de TP
		%elle ne démarre pas
		%elle affiche des erreur de réseau
		%elle m'insulte en allemand

\subsection{Utilisation d'un laptop}
	%J'utilise mon laptop
		
% après voir ecrit cette journée type et les différentes erreurs, il est pratiquement toujours possible de rediriger vers l'une ou l'autre des section suivantes.

\section{Dépannage d'un PC de gestion de l'AIP}
%on ecrit une subsection par etape de dépannage
%l'idée est qu'une fois que le mec à suivi la subsection 1, il passe à la subsection 2 si ca ne marche toujours pas.

\section{Dépannage d'une Plateforme de TP Mobile}
%on ecrit une subsection par etape de dépannage
%l'idée est qu'une fois que le mec à suivi la subsection 1, il passe à la subsection 2 si ca ne marche toujours pas.

\section{Dépannage d'un PC personnel et laptop}
%on ecrit une subsection par etape de dépannage
%l'idée est qu'une fois que le mec à suivi la subsection 1, il passe à la subsection 2 si ca ne marche toujours pas.





\end{document}
